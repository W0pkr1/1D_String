\documentclass{article}[18pt]
\usepackage[utf8]{inputenc}
\usepackage[T1]{fontenc}
\usepackage[magyar]{babel}
\usepackage[top=1.5cm,bottom=1.5cm,left=1.5cm,right=1.5cm]{geometry}
\usepackage{amssymb}
\usepackage{amsmath}
\usepackage{textcomp}
\usepackage{graphicx}
\graphicspath{ {images/} }
\usepackage{float}
\usepackage{colortbl}
\usepackage[demo]{graphicx}
\usepackage{subfig}
\usepackage{natbib}
\usepackage{graphicx}
\usepackage{hyperref}
\begin{document}
\begin{titlepage}
\begin{center}
\vspace*{1cm}
 
\Huge
\textit{Számítógépes szimulációk
}
 
\LARGE
\vspace{2cm}
\textbf{1D Húr}
\vfill
    
 
\vspace{0.8cm}
Godó Dániel\\
W0PKR1\\
2021.04.25\\
 
\end{center}
\end{titlepage}
\newpage
 \tableofcontents
 \newpage
\section{Bevezető}
Ebben a beadandóban egy 1 dimenziós húr mozgását vizsgáltuk. A mozgás leírásához a következő egyenletet használtuk:

\begin{equation}
    y_{i,j+1}=2y_{i,j}-y_{i,j-1}+(\frac{c}{c'})^{2}(y_{i+1,j}+y_{i-1,j}-2y_{i,j})
    \label{visibility}
\end{equation}
ahol $j$ jelöli az időbeli mozgást, $i$ a poziciót és $  c'= \frac{\delta x}{\delta t}$  ami a staibilitást jelöli. A rendszert pendített állapotból szimuláltuk a rendszer két szélén rögzített peremfeltetétel mellett.

\section{Első feladat}
Az első feladat sroán elkellett készíteni a szimulációt és ábrát illetve animációt kellett készíteni a húr mozgásárol.

\begin{figure}[h]
    \centering
    \float{{\includegraphics[width=15cm]{letöltés.png} }}%
\end{figure}

Az ábrákból jól látszik, hogy a szimuláció nem mutat semmi rendelleneséget és az animáción sem tapasztalható semmi furcsa viselkedés. \href{https://gph.is/g/4D8ppYj}{Az első animáció linkjéhez ide kattints}.

\section{Második feladat}
A második feladat során az analitikus megoldást kellett összehasonlítani a numerikus szimulációval ez számomra nem sikerült, mert nem volt időm rá.


\section{Harmadik feladat}
A harmadik feladat során a húr sebességét kellet megbecsülni. Az általam szimulált első húr terjedési sebessége $c=83,33$ lépés/időegységre becsültem. Az alapján, hogy mekkora a T periódus ideje és hogy mekkora maga a húr hossza.
\newpage
\section{Negyedik feladat}
A negyedik feladatsorán felkellett térképeznünk a rendszer viselkedését különböző $\frac{c}{c'}$ értékekre úgy, hogy $c<=c'$. Ezeket a következő értékekre vizsgáltam meg
\begin{itemize}
    \item $\frac{c}{c'}$ = 0.2 \href{https://gph.is/g/amMwwBQ}{Az animációhoz kattints ide}.
    \item $\frac{c}{c'}$ = 0.4 \href{https://gph.is/g/ZygmmLO}{Az animációhoz kattints ide}.
    \item $\frac{c}{c'}$ = 0.6 \href{https://gph.is/g/4M6j2WA}{Az animációhoz kattints ide}.
    \item $\frac{c}{c'}$ = 0.8 \href{https://gph.is/g/E3XoDgL}{Az animációhoz kattints ide}.
\end{itemize}
változtatva a $\frac{c}{c'}$ paraméter értékeit a rendszer egészen stabilnak tekinthető az egyedüli változást a rendszer sebességének lefolyásában tapasztaltam.

\section{Ötödik feladat}

Az ötödik feladatsorán különböző kezdőfeltételek mellett kellet vizsgálnunk a rendszert.
A rendszerben 2 kitérési állapotból indítottam elsőnek messziről. \href{https://gph.is/g/ZP8A2Jr}{Az animációhoz ide kattints}.
Majd ezt megvizsgáltam közeli módusokkal is. \href{https://gph.is/g/aX8M2WO}{Az animációhoz ide kattints}.
Itt jól láthatóak a hullámfrontok és az elvárt viselkedést kaptam az animációkból.

\section{Összefoglalás}
A feladatokat többségét sikeresnek mondhatom kivétel a második feladat megoldását, mert arra nem jutott időm. Ahol nem tüntettem fel a $\frac{c}{c'}$ értékét ott 0.8-al szimuláltam és a teljes feladatmegoldás során a húr hosszát $L$ 50-nek vettem. Az alábbi \href{https://github.com/W0pkr1/1D_String.git}{LINKEN}, pedig ellehet érni a szimuláció program kódját és egyéb fájlokat amiket szinétn feltöltök a kooplexre.

\end{document}


